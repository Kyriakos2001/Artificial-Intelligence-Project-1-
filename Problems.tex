\documentclass{article}
\usepackage{fontspec}
\usepackage{amsmath}
\usepackage{amssymb}
\usepackage{hyperref}

\setmainfont{Times New Roman}

\title{Τεχνητή Νοημοσύνη}
\author{Κυριάκος Λάμπρος Κιουράνας}
\date{ΑΜ: 1115201900238}

\begin{document}

\maketitle

\section*{Πρόβλημα 2}

Σε αυτό το πρόβλημα, εξετάζουμε τις τρεις διαφορετικές στρατηγικές αναζήτησης (BFS, DFS, IDS) σε ένα δέντρο με παράγοντα διακλάδωσης \( b = 3 \) και στόχο σε βάθος \( d = 4 \).

Ας συνοψίσουμε τα αποτελέσματα που βρήκαμε:

\begin{enumerate}
    \item \textbf{Αναζήτηση Πρώτα Κατά Πλάτος (BFS)}: Επεκτείνει όλους τους κόμβους μέχρι το βάθος 4, δηλαδή 121 κόμβους.
    \item \textbf{Αναζήτηση Πρώτα Κατά Βάθος (DFS)}:
    \begin{itemize}
        \item \textbf{Καλύτερη Περίπτωση}: Εάν ο στόχος είναι ο πρώτος κόμβος που θα συναντηθεί στο βάθος 4, η DFS επεκτείνει μόνο 5 κόμβους.
        \item \textbf{Χειρότερη Περίπτωση}: Εάν ο στόχος είναι ο τελευταίος κόμβος ή δεν υπάρχει, τότε επεκτείνονται 121 κόμβοι (όλοι οι κόμβοι μέχρι το βάθος 4), όπως στο BFS.
    \end{itemize}
    \item \textbf{Αναζήτηση με Επαναληπτική Εκβάθυνση (IDS)}: Απαιτεί 179 επεκτάσεις κόμβων συνολικά, λόγω των πολλαπλών αναζητήσεων DFS με αυξανόμενο βάθος.
\end{enumerate}

Με βάση τα παραπάνω:
\begin{itemize}
    \item Η \textbf{BFS} εγγυάται την εύρεση του στόχου στον μικρότερο δυνατό αριθμό βημάτων (εάν υπάρχει στο συγκεκριμένο βάθος), αλλά επεκτείνει όλους τους κόμβους μέχρι το βάθος του στόχου.
    \item Η \textbf{DFS} είναι πιο αποδοτική στην καλύτερη περίπτωση, όμως μπορεί να καταλήξει σε μεγάλο αριθμό επεκτάσεων στη χειρότερη περίπτωση.
    \item Η \textbf{IDS} παρέχει μια πιο ισορροπημένη λύση, συνδυάζοντας τα πλεονεκτήματα του BFS (εύρεση του στόχου σε ελάχιστο βάθος) και του DFS (μικρή χρήση μνήμης), αλλά επεκτείνει περισσότερους κόμβους λόγω των επαναλήψεων.
\end{itemize}

\section*{Πρόβλημα 3}

Για το πρόβλημα εύρεσης βέλτιστης διαδρομής σε ένα πλέγμα \( 10 \times 10 \) με τον αλγόριθμο A*, χρησιμοποιώντας μια παραδεκτή ευρετική συνάρτηση, ακολουθούμε τα εξής βήματα:

\subsection*{1. Ευρετική Συνάρτηση}

Η ευρετική συνάρτηση που επιλέχθηκε είναι:

\[
h(n) = \frac{\text{Απόσταση Manhattan από τον κόμβο } n \text{ στον στόχο}}{2}
\]

Αυτή είναι \textbf{παραδεκτή} (admissible) επειδή δεν υπερεκτιμά το πραγματικό κόστος (δηλαδή το κόστος της διαδρομής από το σημείο εκκίνησης ως τον στόχο).

\subsection*{2. Βήματα του Αλγορίθμου A*}

\begin{itemize}
    \item \textbf{Αρχικοποίηση}: Ορίζουμε τον κόμβο εκκίνησης ``S'' με αρχικό κόστος \( g(S) = 0 \) και υπολογίζουμε \( f(S) = g(S) + h(S) \).
    \item \textbf{Επέκταση Κόμβων}: Σε κάθε βήμα, επιλέγουμε τον κόμβο με το μικρότερο \( f(n) \), όπου \( f(n) = g(n) + h(n) \). Εξετάζουμε τους γειτονικούς κόμβους και προσθέτουμε το αντίστοιχο κόστος κίνησης.
    \item \textbf{Εύρεση Διαδρομής}: Συνεχίζουμε την επέκταση μέχρι να φτάσουμε στον στόχο ``G'' (τερματικό σημείο). Η βέλτιστη διαδρομή είναι αυτή με το μικρότερο συνολικό κόστος.
\end{itemize}

\subsection*{3. Παραδείγματα Ευρετικών Συναρτήσεων}

\begin{itemize}
    \item \textbf{Απόσταση Manhattan}:
    \[
    h_1(n) = \text{Απόσταση Manhattan από } n \text{ στον στόχο}
    \]
    Είναι παραδεκτή, καθώς αποτελεί υποεκτίμηση του κόστους.
    \item \textbf{Μισή Απόσταση Manhattan}:
    \[
    h_2(n) = \frac{\text{Απόσταση Manhattan από } n \text{ στον στόχο}}{2}
    \]
    Επίσης παραδεκτή και ενδέχεται να είναι πιο ακριβής σε πλέγματα με κόστη κίνησης, κατευθύνοντας τον A* πιο κοντά στη βέλτιστη διαδρομή.
\end{itemize}

\subsection*{4. Ανάλυση με Εναλλακτικές Στρατηγικές Αναζήτησης}

\begin{itemize}
    \item \textbf{Αναζήτηση Πρώτα σε Πλάτος (BFS)}: Εξετάζει κάθε επίπεδο πλήρως, χωρίς να χρησιμοποιεί ευρετική. Συνεπώς, επεκτείνει περισσότερους κόμβους, ειδικά σε μεγαλύτερα πλέγματα.
    \item \textbf{Αναζήτηση Πρώτα σε Βάθος (DFS)}: Πηγαίνει όσο το δυνατόν βαθύτερα πριν επιστρέψει, κάτι που δεν εγγυάται βέλτιστη διαδρομή σε πλέγμα με κόστη.
    \item \textbf{Αναζήτηση με Επαναληπτική Εκβάθυνση (IDS)}: Διευρύνει το βάθος διαδοχικά, αλλά δεν αξιοποιεί ευρετική πληροφόρηση, με αποτέλεσμα να επεκτείνει περισσότερους κόμβους σε μεγάλα πλέγματα.
    \item \textbf{Απληστική Αναζήτηση Πρώτα στο Καλύτερο}: Επεκτείνει μόνο με βάση την ευρετική τιμή, κάτι που δεν εγγυάται βέλτιστη διαδρομή.
    \item \textbf{Αναζήτηση A*}: Συνδυάζει το πραγματικό κόστος \( g(n) \) και την ευρετική \( h(n) \) για πιο αποδοτική εύρεση βέλτιστης διαδρομής με το λιγότερο δυνατό αριθμό επεκτάσεων κόμβων.
\end{itemize}

\section*{Πρόβλημα 4}

Στο πρόβλημα αυτό εξετάζουμε την αμφίδρομη αναζήτηση και τις διαφορετικές παραλλαγές της, αξιολογώντας την πληρότητα, τη βέλτιστη απόδοση και τους τρόπους ελέγχου συνάντησης των δύο αναζητήσεων.

\subsection*{Αμφίδρομη Αναζήτηση}

Η αμφίδρομη αναζήτηση περιλαμβάνει δύο ταυτόχρονες αναζητήσεις:

\begin{itemize}
    \item Μία αναζήτηση \textbf{προς τα εμπρός} από την αρχική κατάσταση.
    \item Μία αναζήτηση \textbf{προς τα πίσω} από την κατάσταση στόχου.
\end{itemize}

Στόχος είναι οι δύο αναζητήσεις να συναντηθούν κάπου ενδιάμεσα, μειώνοντας το συνολικό χρόνο αναζήτησης σε σχέση με μια μονοκατευθυντική αναζήτηση.

\subsection*{Ερώτημα (α): BFS προς τα εμπρός και Αναζήτηση Περιορισμένου Βάθους προς τα πίσω}

\begin{itemize}
    \item \textbf{Πληρότητα}:
    \begin{itemize}
        \item Η \textbf{BFS} είναι πλήρης, καθώς εξερευνά όλους τους κόμβους σε κάθε επίπεδο.
        \item Η \textbf{αναζήτηση περιορισμένου βάθους} δεν είναι πλήρης, καθώς μπορεί να αποτύχει να βρει λύσεις που βρίσκονται πέρα από το όριό της.
        \item \textbf{Συμπέρασμα}: Ο συνδυασμός δεν είναι πλήρης, διότι η αναζήτηση προς τα πίσω μπορεί να μην συναντήσει την αναζήτηση προς τα εμπρός.
    \end{itemize}
    \item \textbf{Βέλτιστοτητα}:
    \begin{itemize}
        \item Η \textbf{BFS} είναι βέλτιστη όταν όλα τα κόστη είναι ίσα.
        \item Η \textbf{αναζήτηση περιορισμένου βάθους} δεν είναι βέλτιστη.
        \item \textbf{Συμπέρασμα}: Ο συνδυασμός δεν είναι βέλτιστος.
    \end{itemize}
\end{itemize}

\subsection*{Ερώτημα (β): IDS προς τα εμπρός και Αναζήτηση Περιορισμένου Βάθους προς τα πίσω}

\begin{itemize}
    \item \textbf{Πληρότητα}:
    \begin{itemize}
        \item Η \textbf{IDS} είναι πλήρης, καθώς αυξάνει το όριο βάθους επαναληπτικά.
        \item Η \textbf{αναζήτηση περιορισμένου βάθους} δεν είναι πλήρης.
        \item \textbf{Συμπέρασμα}: Ο συνδυασμός δεν είναι πλήρης λόγω του περιορισμένου βάθους προς τα πίσω.
    \end{itemize}
    \item \textbf{Βέλτιστοτητα}:
    \begin{itemize}
        \item Η \textbf{IDS} είναι βέλτιστη για ίσα κόστη.
        \item Η \textbf{αναζήτηση περιορισμένου βάθους} δεν είναι βέλτιστη.
        \item \textbf{Συμπέρασμα}: Ο συνδυασμός δεν είναι βέλτιστος.
    \end{itemize}
\end{itemize}

\subsection*{Ερώτημα (γ): A* προς τα εμπρός και Αναζήτηση Περιορισμένου Βάθους προς τα πίσω}

\begin{itemize}
    \item \textbf{Πληρότητα}:
    \begin{itemize}
        \item Ο \textbf{A*} είναι πλήρης με παραδεκτή ευρετική συνάρτηση.
        \item Η \textbf{αναζήτηση περιορισμένου βάθους} δεν είναι πλήρης.
        \item \textbf{Συμπέρασμα}: Ο συνδυασμός δεν είναι πλήρης, λόγω της μη πληρότητας της αναζήτησης προς τα πίσω.
    \end{itemize}
    \item \textbf{Βέλτιστοτητα}:
    \begin{itemize}
        \item Ο \textbf{A*} είναι βέλτιστος με παραδεκτή και συνεπή ευρετική.
        \item Η \textbf{αναζήτηση περιορισμένου βάθους} δεν είναι βέλτιστη.
        \item \textbf{Συμπέρασμα}: Ο συνδυασμός δεν είναι βέλτιστος.
    \end{itemize}
\end{itemize}

\subsection*{Ερώτημα (δ): A* προς τα εμπρός και A* προς τα πίσω}

\begin{itemize}
    \item \textbf{Πληρότητα}:
    \begin{itemize}
        \item Και οι δύο αναζητήσεις A* είναι πλήρεις, με παραδεκτή ευρετική.
        \item \textbf{Συμπέρασμα}: Ο συνδυασμός είναι πλήρης.
    \end{itemize}
    \item \textbf{Βέλτιστοτητα}:
    \begin{itemize}
        \item Ο \textbf{A*} είναι βέλτιστος με παραδεκτή και συνεπή ευρετική.
        \item \textbf{Συμπέρασμα}: Ο συνδυασμός είναι βέλτιστος.
    \end{itemize}
\end{itemize}

\subsection*{Έλεγχος Συνάντησης των Δύο Αναζητήσεων}

Για την αποδοτική ανίχνευση της συνάντησης των δύο αναζητήσεων:
\begin{itemize}
    \item \textbf{Χρησιμοποιούμε σύνολα (sets)}: Κρατάμε ένα σύνολο με τους κόμβους που έχει επισκεφτεί η κάθε αναζήτηση.
    \item \textbf{Έλεγχος κατά την επέκταση}: Όταν επεκτείνουμε έναν κόμβο σε μία αναζήτηση, ελέγχουμε αν αυτός ο κόμβος βρίσκεται στο σύνολο της άλλης αναζήτησης.
    \item \textbf{Συνάντηση}: Αν βρεθεί κοινός κόμβος, οι αναζητήσεις συναντήθηκαν και μπορούμε να συνθέσουμε τη διαδρομή.
\end{itemize}

\end{document}
